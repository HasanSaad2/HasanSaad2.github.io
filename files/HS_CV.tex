\documentclass[11pt]{amsart}

\usepackage{fullpage}
\usepackage{url}
\usepackage{hyperref}
\usepackage{amssymb}
\usepackage{parskip}
\usepackage{cite}
\usepackage{amsmath,mathtools}
\usepackage[mathscr]{eucal}
\usepackage{tikz}
\usepackage{ytableau}
\usepackage{amsthm}
\usepackage{xcolor}
\usepackage[T1]{fontenc}
\usepackage{enumitem}

\newtheorem{theorem}{Theorem}
\newtheorem*{corollary}{Corollary}
\newtheorem{question}[theorem]{Question}
\newtheorem*{conjecture}{Conjecture}
\theoremstyle{remark}
\newtheorem*{remark}{Remark}
\newtheorem*{example}{Example}

\newcommand{\Q}{\mathbb{Q}}
\newcommand{\G}{\mathbb{G}}
\newcommand{\F}{\mathbb{F}}
\newcommand{\HH}{\mathbb{H}}
\newcommand{\C}{\mathbb{C}}
\newcommand{\Z}{\mathbb{Z}}
\newcommand{\Zhat}{\widehat Z}
\newcommand{\Zhathat}{\widehat{\vphantom{\rule{5pt}{10pt}}\smash{\widehat{Z}}\,}\!}


\newcommand{\resheading}[1]{\medskip
    \colorbox{lightgray}{\begin{minipage}{\dimexpr\textwidth-2\fboxsep\relax}\small 
                        \textbf{#1 \vphantom{p\^{E}}}
                      \end{minipage}}
                            \par\medskip}
\newcommand{\ressubheading}[3]{
\begin{tabular*}{\textwidth}{l @{\extracolsep{\fill}} r}
    \textsc{\textbf{#1}} & \textsc{\textit{[#2]}} \\
\end{tabular*}\vspace{-8pt}}

\title{\large{Curriculum Vitae}}
\author{\large{Hasan Saad}}
\begin{document}

\textbf{ }


\maketitle 


\large{Office Address: 123 Kerchof Hall, Department of Mathematics, University of Virginia}

\large{Email: \href{mailto:hs7gy@virginia.edu}{hs7gy@virginia.edu} \hfill Website: \href{https://hasansaad2.github.io/}{hasansaad2.github.io} \hfill Phone:+1 (434) 227-7173}

\vspace{.1in}


\large{\textbf{\textsc{\textsc{Education}}}

University of Virginia $\circ$ 2020 -- Spring 2024 Expected \\
Ph.D. in Mathematics\\ Advisor: Ken Ono \\
Thesis: \textit{On the Distributions of Point Counts on Hypergeometric Varieties}

\vspace{.1in}

American University of Beirut $\circ$ 2018 -- December 2020\footnote{I was accepted to the Ph.D. program at UVA before my last semester, and therefore I did not complete my studies.} \\
M.S. in Mathematics\\ Advisor: Wissam Raji

\vspace{.1in}

Lebanese University $\circ$ 2015 -- 2018 \\
B.S. in Mathematics

\vspace{.1in}


\large{\textbf{\textsc{Research Publications}}}
\begin{enumerate}[label=\arabic*.]
\item K. Ono, H. Saad and N. Saikia, \textit{Distribution of values of Gaussian hypergeometric functions.}  Pure and Applied Mathematics Quarterly, Special Issue for Don Zagier’s 70th birthday, {\bf 19}, no. 1 (2023), 371-407. 
\item H. Saad, \textit{Explicit Sato--Tate type distribution for a family of $K3$ surfaces.} Forum Mathematicum {\bf 35}, no. 4, 1105-1132.
\item K. Ono and H. Saad, \textit{Some Eichler--Selberg Trace Formulas.} The Hardy Ramanujan Journal {\bf 45}, 94-107.
\item Y. Huang, K. Ono, and H. Saad \textit{Matrix points on special varieties over finite fields.} To appear in Contemporary Mathematics, American Mathematical Society.
\item K. Satoshi and H. Saad \textit{On Matrices Arising in Finite Field Hypergeometric Functions.} Preprint. \texttt{https://arxiv.org/abs/2312.02890}.

\end{enumerate}

\large{\textbf{\textsc{Conference and Seminar Talks}}}

\bgroup
\def\arraystretch{1.3}
\begin{tabular}{ c c c }
2024 & Jan &  \hspace{-2in}Joint Mathematics Meetings, San Francisco \\
     &     & AMS Special Session on Mock modular forms, physics, and applications \\
     &     & \hspace{-0.80in}\textit{Automorphic Forms and Point Distributions on $K3$ Surfaces} \\
	
\end{tabular}


\bgroup
\def\arraystretch{1.3}
\begin{tabular}{ c c c }
	& Feb & \hspace{-3.1in}Clifford Lectures, Tulane University \\
	&     & \hspace{-2.64in}\textit{Point Distributions on Algebraic Varieties} \\
	& Mar & \hspace{-2.2in} Algebra, Geometry, and Number Theory Seminar\\
	&     & \hspace{-0.9in}\textit{Counting matrix points on hypergeometric varieties over finite fields} \\
2023 & May & \hspace{-1.2in} 35th Automorphic Forms Workshop, Louisiana State University \\
     &     & \hspace{-1.45in}\textit{Determining point distributions on hypergeometric varieties} \\
     & Apr & AMS Special Session on Hypergeometric Functions, $q$-series and Generalizations \\
     &     & \hspace{-0.9in}\textit{Counting matrix points on hypergeometric varieties over finite fields} \\
     & Mar & \hspace{-0.45in}Specialty Seminar in Partition Theory, $q$-Series and Related Topics, MTU \\
     &     & \hspace{-1.4in}\textit{Counting matrix points on curves and surfaces with partitions} \\
     & Feb & \hspace{-0.53in} Number Theory and Combinatorics Seminar, University of Texas at Tyler \\
     &      & \hspace{-1.5in}\textit{Explicit Sato--Tate distributions for hypergeometric varieties} \\
     & Feb  & \hspace{-2.3in} Ramanujan--Serre Seminar, University of Virginia \\
     &      & \hspace{-1.5in}\textit{Counting matrix points on certain varieties over finite fields} \\
     &  Jan & \hspace{-3.13in}Joint Mathematics Meetings, Boston \\
     &      & \hspace{-0.9in}AMS Special Session on Modular Forms, Hypergeometric Functions, \\
     &      & \hspace{-2.5in}Character Sums and Galois Representations I\\
     &      &  \hspace{-1.5in}\textit{Explicit Sato--Tate distributions for hypergeometric varieties} \\
2022 &  Oct & \hspace{-2.38in}Number Theory Seminar, University of Cologne \\
     &      &  \hspace{-1.5in}\textit{Explicit Sato--Tate distributions for hypergeometric varieties} \\
     &  Oct & \hspace{-2.38in}Number Theory Seminar, Vanderbilt University \\
     &      &  \hspace{-1.5in}\textit{Explicit Sato--Tate distributions for hypergeometric varieties} \\
     & Sep  & \hspace{-1.15in} Algebra and Number Theory Seminar, Louisiana State University \\
     &      & \hspace{-1.5in}\textit{Explicit Sato--Tate distributions for hypergeometric varieties} \\
     & Sep  & \hspace{-2.3in}Ramanujan-Serre Seminar, University of Virginia \\
     &      & \hspace{-1.9in}\textit{Sato--Tate type distribution for a family of K3 surfaces} \\
     & Jul  & \hspace{-2.2in}Hong Kong University Number Theory Days 2022 \\
     &      & \hspace{-1.4in}\textit{Distribution of Values of Gaussian Hypergeometric Functions}\\
2021 & Dec  & \hspace{-2.8in} Analysis Seminar, Stony Brooks University \\
     &      & \hspace{-1.42in}\textit{Distribution of Values of Gaussian Hypergeometric Functions} \\
     &  Nov & \hspace{-2in}Mathematics Seminar, American University of Beirut \\
     &      & \hspace{-1.42in}\textit{Distribution of Values of Gaussian Hypergeometric Functions}\\
     &  Oct & \hspace{-2.7in} Number Theory Seminar, Boston University \\ 
     &      & \hspace{-1.42in}\textit{Distribution of Values of Gaussian Hypergeometric Functions} \\
     &  Aug & \hspace{-2.4in} Number Theory Seminar, University of Virginia \\
     &      & \hspace{-1.1in} \textit{Frobenius trace distributions for Gaussian hypergeometric varieties}
\end{tabular}
\egroup

\

\large{\textbf{\textsc{Additional Conference Participation}}}
\begin{itemize}
\item May 2022: 100 Years of Mock Theta Functions, Vanderbilt University
\end{itemize}

\

\large{\textbf{\textsc{Mentorship}}}
\begin{itemize}
\item Summer 2023: Lead Mentor for the University of Virginia REU in Number Theory. \ Mentored a project on Sato--Tate type distributions for matrix points on varieties. 
\item Summer 2022: Mentor for the University of Virginia REU in Number Theory. \\ Advised a project on Sato--Tate analogue for some $K3$ surfaces.
\end{itemize}

\
 
\large{\textbf{\textsc{Teaching}}}

\bgroup
\def\arraystretch{1.3}
\begin{tabular}{ c c c }
2023 & Fall & \hspace{-0.1in}\hspace{-0.65in}\hspace{-0.85in}Instructor of record for MATH 1320 (Calculus II), U.Va. \\
2022 & Fall &  \hspace{-0.1in}\hspace{-0.65in}Instructor of record for MATH 1210 (A Survey of Calculus I), U.Va. \\
     & Spring & \hspace{-0.1in}\hspace{-0.65in}Instructor of record for MATH 1210 (A Survey of Calculus I), U.Va. \\
2021 & Fall &  \hspace{-0.1in}\hspace{-0.65in}Instructor of record for MATH 1210 (A Survey of Calculus I), U.Va.  \\
2021 & Summer & \hspace{-0.1in}\hspace{-0.65in}\hspace{-1in}Teaching Assistant for MATH 1310 (Calculus I), U.Va. \\
     &        & \hspace{-0.1in}\hspace{-0.65in}\hspace{-0.03in}Teaching Assistant for MATH 1220 (A Survey of Calculus II), U.Va. \\
     &        & \hspace{-0.1in}\hspace{-0.65in}\hspace{-0.95in}Teaching Assistant for MATH 1320 (Calculus II), U.Va. \\
2020 &  Fall   & Teaching Assistant for MATH 201 (Calculus and Analytic Geometry III), AUB.\\
2019 & Spring   & \hspace{-0.1in}Teaching Assistant for MATH 101 (Calculus and Analytic Geometry I), AUB. \\
     & Fall     & \hspace{-0.1in}Teaching Assistant for MATH 101 (Calculus and Analytic Geometry I), AUB.

     


\end{tabular}
\egroup

\

\large{\textbf{\textsc{Data Science and Machine Learning}}}

In addition to my research and teaching experiences, I have gained experience in machine learning through the Erdős Institute Data Science Boot Camp which I outline in this page.

\normalsize{\textbf{\textsc{Detecting Images Generated by Neural Networks}}}

\begin{itemize}
	
	\item {\bf Project Description}
	
	The recent advances in deep learning, neural networks, and the hardware to support it have provided fertile ground for creating fake images. This new technology, if left unchallenged, creates a risk in multiple areas, including journalism, law enforcement, and knowledge itself.
	
	We tackle this problem by constructing two multi-classification models (single-channel and dual-channel) to discern between real images and those which are generated by AI, and to determine which generative algorithm was used. Our model is trained on a publicly available dataset of approximately 90000 images. This dataset contains real images as well as images generated by 13 different CNN-based generative algorithms.
	
	\vspace{0.5in}
	
	\item {\bf Model Description}
	
	Our dual-channel model operates in two stages. 
	
	In the first stage, a copy of the image passes through each channel after undergoing filtration. In the first channel, a high pass filter using Gaussian blur is applied. In the second channel, a Discrete Cosine Transform is applied. 
	
	In the second stage, after passing through multiple convolutional and pooling layers, the two channels are connected. The connecting channel is fully connected and has a convolutional and a pooling layer.
	
	Finally, the output layer consists of softmax functions to determine the probabilities of each model being the generating model and the probability of the image being real.
	
	\item {\bf Benchmarks}
	
	To evaluate this model, we used multiclass precision and recall metrics. Due to the non-homogeneity of data, detection performance varied according to the model generating the image. For purely detecting whether an image is real or not, we have a precision of approximately 90\% and a recall of approximately 93\%.
	
	\item {\bf Certificate}
	
	The project outlined here ranked as a top-5 project among approximately 40 teams. Through this project, I obtained a Data Science certificate from the Erdős Institute. This certificate can be found at
	\href{https://www.erdosinstitute.org/certificates/fall-2023/data-science-boot-camp/hasan-saad}{https://www.erdosinstitute.org/certificates/fall-2023/data-science-boot-camp/hasan-saad}

	\item {\bf Github Link}
	
	The Github repository containing this project can be found at \href{https://github.com/Alina-Beaini/AIvsReal}{https://github.com/Alina-Beaini/AIvsReal}
	
	\item {\bf Hugging Face}
	
	Furthermore, we created a Gradio interface to showcase the model. This interface was hosted on Hugging Face. It can be found at  \href{https://huggingface.co/spaces/AlinaBeaini/AIvsReal}{https://huggingface.co/spaces/AlinaBeaini/AIvsReal}
	
\end{itemize}

\large{\textbf{\textsc{References}}}

\begin{itemize}
	\item {Ken Ono  \\
		Marvin Rosenblum Professor at the University of Virginia \\
		\large{Email: \href{mailto:ko5wk@virginia.edu}{ko5wk@virginia.edu}}}
	\item {{Wen-Ching Winnie Li \\
			Distinguished Professor at The Pennsylvania State University} \\
		\large{Email: \href{mailto:wli@math.psu.edu}{wli@math.psu.edu}}}
	\item {{Kathrin Bringmann \\
			W3 Professor (Full Professor) at University of Cologne} \\
		\large{Email: \href{mailto:kbringma@math.uni-koeln.de}{kbringma@math.uni-koeln.de}}}
	\item {{ Jim Rolf - Teaching Reference  \\
			Professor, General Faculty at the University of Virginia} \\
		\large{Email: \href{mailto:jsr2pz@virginia.edu}{jsr2pz@virginia.edu}}}
	
\end{itemize}





\end{document}
